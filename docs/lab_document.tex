\documentclass[11pt,a4paper]{article}
\usepackage[utf8]{inputenc}
\usepackage[margin=1in]{geometry}
\usepackage{graphicx}
\usepackage{listings}
\usepackage{xcolor}
\usepackage{hyperref}
\usepackage{fancyhdr}
\usepackage{titlesec}
\usepackage{enumitem}
\usepackage{booktabs}
\usepackage{tabularx}
\usepackage{float}

% Colors
\definecolor{codegreen}{rgb}{0,0.6,0}
\definecolor{codegray}{rgb}{0.5,0.5,0.5}
\definecolor{codepurple}{rgb}{0.58,0,0.82}
\definecolor{backcolour}{rgb}{0.95,0.95,0.92}
\definecolor{primaryblue}{RGB}{88,166,255}

% Code listing style
\lstdefinestyle{mystyle}{
    backgroundcolor=\color{backcolour},
    commentstyle=\color{codegreen},
    keywordstyle=\color{magenta},
    numberstyle=\tiny\color{codegray},
    stringstyle=\color{codepurple},
    basicstyle=\ttfamily\footnotesize,
    breakatwhitespace=false,
    breaklines=true,
    captionpos=b,
    keepspaces=true,
    numbers=left,
    numbersep=5pt,
    showspaces=false,
    showstringspaces=false,
    showtabs=false,
    tabsize=2,
    frame=single
}
\lstset{style=mystyle}

% Header/Footer
\pagestyle{fancy}
\fancyhf{}
\fancyhead[L]{CPEG 460 - Computer Networks}
\fancyhead[R]{Lab: SecureNet DC}
\fancyfoot[C]{\thepage}

% Title formatting
\titleformat{\section}{\Large\bfseries\color{primaryblue}}{\thesection}{1em}{}
\titleformat{\subsection}{\large\bfseries}{\thesubsection}{1em}{}

\begin{document}

% Title Page
\begin{titlepage}
    \centering
    \vspace*{2cm}

    {\Huge\bfseries SecureNet DC\\[0.5cm]}
    {\Large Building and Defending a Software-Defined\\Data Center Network\\[2cm]}

    {\large\textbf{Lab Experiment Document}\\[0.5cm]}

    \vspace{1cm}

    {\Large CPEG 460 - Computer Networks\\[0.3cm]}
    {\large Fall 2025\\[0.3cm]}
    {\large Dr. Mohammad Shaqfeh\\[2cm]}

    \vspace{2cm}

    \begin{tabular}{ll}
        \textbf{Duration:} & 4-5 hours \\
        \textbf{Difficulty:} & Advanced \\
        \textbf{Prerequisites:} & TCP/IP, Python, Linux basics \\
    \end{tabular}

    \vfill

    {\small Hamad Bin Khalifa University\\College of Science and Engineering}
\end{titlepage}

\tableofcontents
\newpage

% Introduction
\section{Introduction}

\subsection{Overview}
This lab introduces Software-Defined Networking (SDN) through a comprehensive data center simulation using Mininet. You will explore a Fat-Tree topology, implement SDN controller features, experience DDoS attack detection, and configure Quality of Service (QoS) policies.

\subsection{Learning Objectives}
Upon completing this lab, you will be able to:
\begin{itemize}
    \item Understand Fat-Tree data center topology and its advantages
    \item Program an SDN controller using the Ryu framework
    \item Implement load balancing using OpenFlow
    \item Detect and mitigate DDoS attacks in an SDN environment
    \item Configure QoS policies for traffic prioritization
    \item Analyze network performance using standard tools
\end{itemize}

\subsection{Background}
Software-Defined Networking separates the control plane (decision-making) from the data plane (packet forwarding). The SDN controller makes forwarding decisions and installs flow rules on switches via the OpenFlow protocol.

The Fat-Tree topology is widely used in data centers due to its:
\begin{itemize}
    \item Full bisection bandwidth
    \item Multiple paths for fault tolerance
    \item Scalable, regular structure
\end{itemize}

\newpage

% Part A
\section{Part A: Environment Setup (30 minutes)}

\subsection{Prerequisites}
This lab requires WSL2 (Windows Subsystem for Linux 2) with Ubuntu. Ensure you have:
\begin{itemize}
    \item Windows 10/11 with WSL2 enabled
    \item Ubuntu 20.04+ distribution installed in WSL2
    \item Python 3.11 (recommended)
    \item Mininet 2.3.0+
    \item Open vSwitch 2.13+
\end{itemize}

\subsection{Operating Modes}

The project supports two operating modes depending on your environment:

\begin{table}[H]
\centering
\caption{Operating Mode Comparison}
\begin{tabularx}{\textwidth}{lXX}
\toprule
\textbf{Aspect} & \textbf{Linux Bridge Mode (WSL2)} & \textbf{Full SDN Mode (VM)} \\
\midrule
Switch type & Linux bridges (\texttt{--switch lxbr}) & OVS (\texttt{--switch ovsk}) \\
Controller & Not required & Ryu controller required \\
Flow rules & Not supported & Full OpenFlow support \\
Best for & Traffic generation, detection testing & Complete SDN demonstration \\
Command & \texttt{sudo mn --switch lxbr --topo tree,2} & \texttt{sudo mn --controller remote --topo tree,2} \\
\bottomrule
\end{tabularx}
\end{table}

\textbf{Note:} WSL2 has issues establishing OpenFlow connections. Use Linux Bridge mode for quick testing, or set up a Linux VM (VirtualBox/VMware) for full SDN features.

\subsection{Quick Setup (Recommended)}

The project includes an automated setup script that handles all dependencies:

\begin{lstlisting}[language=bash,caption=Automated Setup]
# Open WSL Ubuntu terminal
cd ~/securenet_dc

# Run the setup script
chmod +x scripts/setup_wsl.sh
./scripts/setup_wsl.sh
\end{lstlisting}

The script automatically:
\begin{itemize}
    \item Creates a Python 3.11 virtual environment
    \item Installs Ryu SDN Framework with correct dependencies
    \item Installs Flask and dashboard dependencies
    \item Applies required patches for eventlet compatibility
    \item Verifies all installations
\end{itemize}

\subsection{Manual Installation (Alternative)}

If you prefer manual installation:

\begin{lstlisting}[language=bash,caption=Manual Installation Commands]
# Update system and install base packages
sudo apt-get update
sudo apt-get install -y mininet openvswitch-switch python3.11 python3.11-venv

# Install network testing tools
sudo apt-get install -y iperf3 hping3 tcpdump

# Create and activate virtual environment
python3.11 -m venv venv
source venv/bin/activate

# Install Python packages (order matters!)
pip install wheel setuptools==57.5.0
pip install --no-build-isolation ryu
pip install flask flask-socketio flask-cors requests scapy matplotlib

# Apply eventlet patch (REQUIRED for Ryu to work)
WSGI_FILE=$(find venv -name "wsgi.py" -path "*/ryu/app/*" | head -1)
sed -i "s/from eventlet.wsgi import ALREADY_HANDLED/ALREADY_HANDLED = b''/" "$WSGI_FILE"
\end{lstlisting}

\subsection{Verify Installation}
\begin{lstlisting}[language=bash,caption=Verification]
# Activate virtual environment
source venv/bin/activate

# Test Mininet
sudo mn --test pingall

# Check OVS
sudo ovs-vsctl show

# Verify Ryu
python -c "import ryu; print(f'Ryu version: {ryu.__version__}')"

# Verify Flask
python -c "import flask; print(f'Flask version: {flask.__version__}')"
\end{lstlisting}

\textbf{Question A.1:} What version of Mininet is installed? What OpenFlow versions does it support?

\newpage

% Part B
\section{Part B: Topology Exploration (45 minutes)}

\subsection{Launch the Network}

\textbf{Recommended: One-Click Launch (Windows)}

Double-click \texttt{START\_SECURENET.bat} in the project folder. This automatically:
\begin{itemize}
    \item Syncs files to WSL
    \item Opens 3 Windows Terminal tabs with all components
\end{itemize}

\textbf{Alternative: Manual Launch (3 WSL terminals)}

\begin{lstlisting}[language=bash,caption=Starting the System Manually]
# TERMINAL 1: Start Mininet with Demo Runner
cd ~/securenet_dc
sudo python3 scripts/run_demo.py

# TERMINAL 2: Start Stats Collector (for detection)
cd ~/securenet_dc
source venv/bin/activate
python3 scripts/network_stats_collector.py

# TERMINAL 3: Start the Dashboard
cd ~/securenet_dc
source venv/bin/activate
python3 dashboard/app.py
\end{lstlisting}

\textbf{Simple Test (without dashboard):}
\begin{lstlisting}[language=bash]
# Quick test with Linux Bridge mode
sudo mn --switch lxbr --topo tree,depth=2,fanout=3
\end{lstlisting}

\subsection{Explore the Topology}
In the Mininet CLI, execute:

\begin{lstlisting}[language=bash,caption=Topology Exploration Commands]
# View all nodes
mininet> nodes

# View network connections
mininet> net

# View detailed information
mininet> dump

# Test connectivity
mininet> pingall
\end{lstlisting}

\subsection{Understanding the Fat-Tree Structure}
The topology has three layers:
\begin{itemize}
    \item \textbf{Core Layer:} 4 switches (c1-c4)
    \item \textbf{Aggregation Layer:} 8 switches (a1-a8), 2 per pod
    \item \textbf{Edge Layer:} 8 switches (e1-e8), 2 per pod
    \item \textbf{Hosts:} 16 hosts (h1-h16), distributed across 4 pods
\end{itemize}

\begin{table}[H]
\centering
\caption{Host Configuration}
\begin{tabular}{lll}
\toprule
\textbf{Hosts} & \textbf{IP Range} & \textbf{Role} \\
\midrule
h1-h4 & 10.0.1.x & Web Servers \\
h5-h6 & 10.0.2.x & Database Servers \\
h7-h12 & 10.0.3.x & Client Hosts \\
h13 & 10.0.4.1 & Attacker \\
h14 & 10.0.4.2 & IDS Monitor \\
h15-h16 & 10.0.5.x & Streaming Servers \\
\bottomrule
\end{tabular}
\end{table}

\textbf{Question B.1:} How many hops does a packet travel from h1 to h4 (same pod)? From h1 to h16 (different pod)?

\textbf{Question B.2:} Calculate the theoretical bisection bandwidth of this k=4 Fat-Tree.

\newpage

% Part C
\section{Part C: SDN Controller Basics (60 minutes)}

\subsection{Understanding OpenFlow}
OpenFlow defines how the controller communicates with switches:
\begin{itemize}
    \item \textbf{Packet-In:} Switch sends packet to controller for decision
    \item \textbf{Flow-Mod:} Controller installs flow rules on switch
    \item \textbf{Packet-Out:} Controller tells switch how to forward a packet
\end{itemize}

\subsection{Examine Flow Tables}
\begin{lstlisting}[language=bash,caption=Viewing Flow Tables]
# View flows on switch s1
mininet> sh ovs-ofctl dump-flows e1

# View flows with statistics
mininet> sh ovs-ofctl dump-flows e1 --stats
\end{lstlisting}

\subsection{L2 Learning Switch Behavior}
Generate traffic and observe flow rule installation:

\begin{lstlisting}[language=bash]
# Generate traffic
mininet> h1 ping -c 5 h7

# Check new flow rules
mininet> sh ovs-ofctl dump-flows e1
\end{lstlisting}

\textbf{Question C.1:} Describe the flow rules installed after the ping. What match fields are used?

\textbf{Challenge C.1:} Modify the controller to log every packet-in event with source and destination MAC addresses.

\newpage

% Part D
\section{Part D: Load Balancer Implementation (45 minutes)}

\subsection{Concept}
The load balancer uses a Virtual IP (VIP) to distribute traffic across multiple backend servers. When a client connects to the VIP, the controller selects a server and rewrites packet headers.

\subsection{Configuration}
\begin{itemize}
    \item VIP: 10.0.0.100
    \item Server Pool: h1 (10.0.1.1), h2 (10.0.1.2), h3 (10.0.1.3), h4 (10.0.1.4)
    \item Algorithm: Round-robin
\end{itemize}

\subsection{Testing}
\begin{lstlisting}[language=bash,caption=Load Balancer Testing]
# Start web servers
mininet> h1 python3 -m http.server 80 &
mininet> h2 python3 -m http.server 80 &
mininet> h3 python3 -m http.server 80 &
mininet> h4 python3 -m http.server 80 &

# Send requests to VIP from client
mininet> h7 curl http://10.0.0.100/

# Send multiple requests
mininet> h7 for i in $(seq 1 8); do curl -s http://10.0.0.100/; done
\end{lstlisting}

\textbf{Question D.1:} How does the controller handle ARP requests for the VIP?

\textbf{Question D.2:} What OpenFlow actions are used to redirect traffic to backend servers?

\textbf{Challenge D.1:} Implement weighted round-robin where h1 gets 50\% of traffic.

\newpage

% Part E
\section{Part E: DDoS Attack and Defense (60 minutes)}

\subsection{Attack Types}
The detection engine monitors for:
\begin{itemize}
    \item \textbf{SYN Flood:} Many TCP SYN packets without completing handshake
    \item \textbf{ICMP Flood:} Excessive ping requests
    \item \textbf{UDP Flood:} High volume of UDP packets
\end{itemize}

\subsection{Detection Algorithm (v3.0)}
The detection system uses TX-based detection to accurately identify attackers:
\begin{itemize}
    \item High RX on switch port = host is \textbf{SENDING} = ATTACKER
    \item High TX on switch port = host is \textbf{RECEIVING} = VICTIM (do not block)
    \item IP-based deduplication prevents duplicate blocking
    \item 5-second cooldown period prevents rapid re-detection
\end{itemize}

\subsection{Detection Thresholds}
\begin{table}[H]
\centering
\caption{DDoS Detection Thresholds}
\begin{tabular}{ll}
\toprule
\textbf{Attack Type} & \textbf{Threshold (pps)} \\
\midrule
SYN Flood & 100 \\
ICMP Flood & 50 \\
UDP Flood & 200 \\
Total Traffic & 500 \\
\bottomrule
\end{tabular}
\end{table}

\subsection{Recommended: Using the Integrated Demo Runner}

The easiest way to run attacks is using the integrated demo runner, which handles Mininet and attacks in one process:

\begin{lstlisting}[language=bash,caption=Integrated Demo Runner]
# Start the integrated demo (from Windows)
# Double-click START_SECURENET.bat

# Or manually in WSL:
cd ~/securenet_dc
source venv/bin/activate
sudo python3 scripts/run_demo.py

# The interactive menu provides:
# 1. ICMP Flood Attack
# 2. SYN Flood Attack
# 3. UDP Flood Attack
# 4. Ping of Death
# 5. Multi-Vector Attack
# 6. Run Demo Scenario (recommended for presentation)
# 7. Test Connectivity
# 8. Show Status
# 9. Open Mininet CLI
# 0. Exit
\end{lstlisting}

\subsection{Alternative: Manual Attack in Mininet CLI}

\begin{lstlisting}[language=bash,caption=DDoS Simulation in Mininet]
# Baseline: Normal traffic from h1
mininet> h1 ping h2

# Launch ICMP flood attack (simple, no extra tools needed)
mininet> h4 ping -f -c 1000 10.0.0.1

# For SYN flood (requires hping3 installed)
mininet> h4 hping3 -S --flood -p 80 10.0.0.1

# For UDP flood
mininet> h4 hping3 --udp --flood -p 53 10.0.0.1

# Alternative: Slow attack for better visibility
mininet> h4 ping -i 0.01 10.0.0.1
\end{lstlisting}

\textbf{Note:} The stats collector logs will show detection alerts like:
\begin{lstlisting}
[DDoS] ATTACK DETECTED on s2-eth2
[DDoS] Attacker: h4 (10.0.0.4)
[DDoS] RX Rate: 15234 pps (threshold: 1000 pps)
[DDoS] BLOCKING host h4 (10.0.0.4)
\end{lstlisting}

\subsection{Capture with Wireshark}
\begin{lstlisting}[language=bash]
# Start capture on edge switch interface
mininet> sh wireshark -i e1-eth1 -k &

# Rerun attack while capturing
\end{lstlisting}

\textbf{Question E.1:} How quickly was the attack detected? What was the packet rate?

\textbf{Question E.2:} What flow rule was installed to block the attacker?

\textbf{Challenge E.1:} Modify the detection threshold and test different values.

\newpage

% Part F
\section{Part F: QoS Configuration (45 minutes)}

\subsection{Traffic Classes}
\begin{table}[H]
\centering
\caption{QoS Queue Configuration}
\begin{tabular}{llll}
\toprule
\textbf{Queue} & \textbf{Class} & \textbf{Min BW} & \textbf{Ports} \\
\midrule
0 & Critical & 50\% & SSH (22), DNS (53) \\
1 & Real-time & 30\% & RTSP (554), Streaming (5001) \\
2 & Interactive & 15\% & HTTP (80, 443) \\
3 & Bulk & 5\% & FTP (21), General \\
\bottomrule
\end{tabular}
\end{table}

\subsection{Testing QoS}
\begin{lstlisting}[language=bash,caption=QoS Testing]
# Start iperf servers
mininet> h15 iperf3 -s &
mininet> h5 iperf3 -s -p 5002 &

# Generate competing traffic
# High-priority (streaming)
mininet> h7 iperf3 -c 10.0.5.1 -u -b 50M -t 30 &

# Low-priority (bulk)
mininet> h8 iperf3 -c 10.0.2.1 -p 5002 -t 30 &

# Observe that streaming maintains throughput
\end{lstlisting}

\textbf{Question F.1:} What DSCP values are assigned to each traffic class?

\textbf{Question F.2:} How does the HTB qdisc enforce bandwidth guarantees?

\textbf{Challenge F.1:} Create a custom QoS policy prioritizing your own application.

\newpage

% Part G
\section{Part G: Performance Analysis (30 minutes)}

\subsection{Automated Benchmarking}
\begin{lstlisting}[language=bash,caption=Performance Testing]
# Throughput test
mininet> iperf h7 h1

# Latency test
mininet> h7 ping -c 20 h1

# Cross-pod latency
mininet> h7 ping -c 20 h16
\end{lstlisting}

\subsection{Expected Results}
\begin{table}[H]
\centering
\caption{Expected Performance Metrics}
\begin{tabular}{lll}
\toprule
\textbf{Test} & \textbf{Metric} & \textbf{Expected Value} \\
\midrule
Same-pod throughput & Bandwidth & ~94 Mbps \\
Cross-pod throughput & Bandwidth & ~90 Mbps \\
Same-pod latency & RTT & ~2-3 ms \\
Cross-pod latency & RTT & ~4-5 ms \\
\bottomrule
\end{tabular}
\end{table}

\textbf{Question G.1:} Compare intra-pod vs inter-pod latency. Explain the difference.

\textbf{Challenge G.1:} Generate performance comparison graphs using the provided tools.

\newpage

% Deliverables
\section{Deliverables}

Submit the following:

\begin{enumerate}
    \item \textbf{Written Answers:} Responses to all questions (A.1, B.1-B.2, C.1, D.1-D.2, E.1-E.2, F.1-F.2, G.1)

    \item \textbf{Wireshark Captures:}
    \begin{itemize}
        \item OpenFlow handshake messages
        \item Attack traffic capture with annotations
        \item Load balancer flow rules
    \end{itemize}

    \item \textbf{Performance Data:}
    \begin{itemize}
        \item Throughput measurements (screenshot or data)
        \item Latency measurements
        \item Generated graphs
    \end{itemize}

    \item \textbf{Challenge Implementations} (for bonus points):
    \begin{itemize}
        \item Code modifications
        \item Test results
    \end{itemize}

    \item \textbf{Analysis Report:} 2-page report discussing:
    \begin{itemize}
        \item Key observations from each part
        \item Challenges encountered
        \item Comparison of expected vs actual results
    \end{itemize}
\end{enumerate}

\subsection{Grading Rubric}

\begin{table}[H]
\centering
\caption{Lab Grading Rubric}
\begin{tabular}{lll}
\toprule
\textbf{Component} & \textbf{Points} & \textbf{Criteria} \\
\midrule
Part A: Setup & 10 & Environment verified, questions answered \\
Part B: Topology & 15 & Exploration complete, calculations correct \\
Part C: SDN Basics & 15 & Flow rules explained, OpenFlow understood \\
Part D: Load Balancer & 15 & VIP tested, NAT mechanism explained \\
Part E: DDoS Defense & 20 & Attack simulated, detection observed \\
Part F: QoS Config & 15 & Traffic classes tested, DSCP understood \\
Part G: Performance & 10 & Measurements taken, analysis provided \\
\midrule
\textbf{Total} & \textbf{100} & \\
\midrule
Challenge Bonus & +20 & Successful challenge implementations \\
\bottomrule
\end{tabular}
\end{table}

\newpage

% Appendix
\section{Appendix: Useful Commands}

\subsection{Mininet Commands}
\begin{lstlisting}[language=bash]
nodes        # List all nodes
net          # Show network topology
dump         # Detailed node info
pingall      # Test all-to-all connectivity
iperf h1 h2  # Bandwidth test
xterm h1     # Open terminal for h1
\end{lstlisting}

\subsection{OpenFlow Commands}
\begin{lstlisting}[language=bash]
ovs-ofctl dump-flows s1     # View flow table
ovs-ofctl dump-ports s1     # View port statistics
ovs-vsctl show              # View OVS configuration
\end{lstlisting}

\subsection{Network Tools}
\begin{lstlisting}[language=bash]
ping -c 10 <ip>            # Latency test
iperf3 -s                  # Start server
iperf3 -c <ip> -t 10       # Bandwidth test
tcpdump -i <interface>     # Packet capture
\end{lstlisting}

\end{document}
